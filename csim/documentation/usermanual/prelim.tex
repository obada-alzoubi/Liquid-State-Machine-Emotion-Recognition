
\subsection{What is \csim?}

\csim is a tool for simulating heterogeneous networks composed of
different model neurons and synapses. This simulator is written in C++
with an MEX interface to Matlab. It is intended to simulate networks
containing up to 10.000 neurons and up to the order of a few millions
of synapses (the actual size depends of course on the amount of RAM
available on your machine).

\subsection{About this Manual}

This manual is intended to describe how to use \csim from the (Matlab)
users point of view. It \emph{does not} try to explain (or give an
introduction to) the type of models which can be simulated with CSIM.
Regarding neural modeling we refer the reader to \cite{DayanAbbott:01}
and \cite{GerstnerKistler:02}.  Furthermore
\href{http://www.mathworks.com}{Matlab} programming knowledge is
assumed.

\html{This manual is also available as \href{usermanual.pdf}{printable
PDF file}.}

\latex{This manual is also available in \href{usermanual.html}{HTML
format}.}


\subsection{Features of the current version}

\begin{description}
  
\item[Different levels of modeling] Different neuron models:
  leaky-integrate-and-fire neurons, compartmental based neurons,
  sigmoidal neurons. Different synapse models: static synapses and a
  certain model of dynamic synapses are available for spiking as well
  as for sigmoidal neurons. Spike time dependent synaptic plasticity
  is also implemented.

  
\item[Easy to use Matlab interface] Since \csim is incorporated into
  Matlab it is not necessary to learn any other script laguage to set
  up the simulation. This is all done with Matlab scripts. Furthermore
  the results of a simulation are directly returned as Matlab arrays
  and hence any plotting and analysis tools available in Matlab can
  easily be applied.
  
\item[Object oriented design] We adopted an object oriented design for
  \csim which is similar to the approaches taken in
  \href{http://www.bbb.caltech.edu/GENESIS/genesis.html}{GENESIS} and
  \href{http://www.neuron.yale.edu}{NEURON}. That is there are objects
  (e.g. a \texttt{LifNeuron} object implements the standard
  leaky-integrate-and-fire model) which are interconnected by means of
  some signal channels. The creation of objects, the connection of
  objects and the setting of parameters of the objects is controlled
  at the level of Matlab scipts whereas the actual simulation is done
  in the fast C++ core.

  
\item[Fast C++ core] Since \csim is implemented in C++ and is not as
  general as GENESIS or NEURON simulations are performed quite fast.
  We also implemented some ideas from event driven simulators like
  \href{http://www.cnl.salk.edu/~arno/spikenet/}{SpikeNet} which
  result on an average speedup of 3 (assuming an average firing rate
  of the neurons of 20\,Hz and short synaptic time constants) compared
  to a standard fixed time step simulation scheme.
  
\item[Runs on Windows and Unix (Linux)] \csim is developed on Linux but also
  runs under Windows XP (we have no more experience with Windows 98 yet, but
  it should also run there) and should in principle run on any platform for
  which Matlab is available.
  
\item[External Interface] There is an external interface which allows \csim to
  communicate with external programs. In this way one can for example control
  the miniature robot
  \href{http://www.k-team.com/robots/khepera/index.html}{Khepera} with \csim.
  This feature is not available in the Windows version.

\end{description}

\subsection{Getting and Installing \csim}

\csim is distributed under the
\href{http://www.gnu.org/copyleft/gpl.html}{GNU General Public
  License} and can be downloaded from
\url{http://www.igi.tugraz.at/csim}.


To install \csim perform the following steps:

\begin{enumerate}
\item Donwload \csim from \url{www.igi.tugraz.at/csim}
\item Unzip the file \texttt{csim-VER.zip} where \texttt{VER} stands
    for the version you have downloaded. 
    
    This will create a subdirectory \texttt{lsm} and \texttt{lsm/csim}

\item Start Matlab and change into the directory \texttt{lsm}
  
\item Run the Matlab script \texttt{install.m}.
  
\item Add the path \texttt{lsm} to the Matlab search path; e. g.
\begin{itemize}
\item  \verb#addpath('/home/jack/lsm')}# or
\item  \verb#addpath('C:\Work\Neuroscience\lsm')#.
\end{itemize}

\item Change into the directory \texttt{lsm/csim/demos} and play
  around with them.

\item Have fun using \csim !

\end{enumerate}


\subsection{Compiling \csim from the sources}
\label{sec:compile}

The instruction in this section are relevant in two cases

\begin{enumerate}

\item[a)] You do not have Linux (x86 architecture) or Windows as an operating
  system.\footnote{The compiled MEX-files (\texttt{csim.mexglx} and
\texttt{csim.dll}) which already come with \csim should run on any
Linux or Windows operationg system respectively.}

\item[b)] You are planning to add your own models written in C++ to CSIM; see
  section~\ref{sec:usermodels} for more information on this topic.

\end{enumerate}

 
%If you do not use Windows or Linux as an operating system \texttt{install.m}
%will try to compile \csim by using Matlabs \texttt{mex} utility.

Currently we support compilation using (GNU) \texttt{make} for Linux/Unix
(which most likly also works for Mac OS X) and \texttt{nmake} (Microsoft
Visual C++) for Windows (tested only with Windows XP). The corresponding
Makefiles are \verb+lsm/csim/src/Makefile+ and
\verb+lsm\csim\src\Makefile.win+.

\subsubsection*{Step by step instructions}

\begin{enumerate}

\item To compile \csim from the sources you must first setup Matlab's mex utility
properly. This can be achieved by issuing the command \verb+mex -setup+ at the
Matlab prompt. For more information on compiling MEX-files see the
\href{http://www.mathworks.com/support/tech-notes/1600/1605.html}{MEX-files
  Guide}.

\item \emph{Optional steps}: \small If you are planning to add your own models
  written in C++ to CSIM and are not using Linux or Windows you have to build
  the \texttt{reggen} tool which comes with \csim (\verb+lsm/develop/reggen+)
  for your platform.\footnote{For Linux and Windows the precompiled
    \texttt{reggen} tool is located in \texttt{lsm/develop/reggen/bin}.}

  \begin{enumerate}
      \item Go to \verb+lsm/develop/reggen+
      \item Run 
      \begin{itemize}
        \item \verb+configure; make;+ for Linux/Unix(/Mac OS)
        \item \verb+make.bat msvc+ for Windows XP with MS Visual C++ 
      \end{itemize}
    \item If this fails read \verb+lsm/develop/reggen/INSTALL+ and try to fix
      the build. In this case you will notice that \verb+reggen+ is a modified
      version of the well known tool doxygen.
  \end{enumerate}

\normalsize 

\item Edit the corresponding Makefile to meet your system configuration.

\item Goto \verb+csim/src+ and run
      \begin{itemize}
        \item \verb+make+ for Linux/Unix(/Mac OS)
        \item \verb+nmake -f Makefile.win+ for Windows (XP)
      \end{itemize}

\end{enumerate}

We do not know whether this instructions work on systems other than Windows XP
with MS Visual C++ and Linux with gcc (3.4 or higher). The
\href{http://www.mathworks.com/access/helpdesk/help/techdoc/matlab_external/matlab_external.shtml}{External
  Interfaces/API} section of the Mathworks web site and the
\href{http://www.mathworks.com/support/tech-notes/1600/1601.shtml}{Tech-Note
  1601} contain valuable information about compiler requirements for compiling
C++ MEX files.

