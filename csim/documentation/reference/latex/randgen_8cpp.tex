\hypertarget{randgen_8cpp}{
\section{randgen.cpp File Reference}
\label{randgen_8cpp}\index{randgen.cpp@{randgen.cpp}}
}


\subsection{Detailed Description}
The core is the function \hyperlink{randgen_8cpp_a15}{uniform\_\-rand()} which is taken from Numerical Recipies in C. It is used to derive the normal distributed generator \hyperlink{randgen_8cpp_a18}{normrnd()}.

{\tt \#include \char`\"{}randgen.h\char`\"{}}\par
\subsection*{Functions}
\begin{CompactItemize}
\item 
double \hyperlink{randgen_8cpp_a15}{uniform\_\-rand} (long $\ast$idum)
\item 
void \hyperlink{randgen_8cpp_a16}{rseed} (long idum)
\item 
double \hyperlink{randgen_8cpp_a17}{unirnd} (void)
\item 
double \hyperlink{randgen_8cpp_a18}{normrnd} (void)
\end{CompactItemize}


\subsection{Function Documentation}
\hypertarget{randgen_8cpp_a18}{
\index{randgen.cpp@{randgen.cpp}!normrnd@{normrnd}}
\index{normrnd@{normrnd}!randgen.cpp@{randgen.cpp}}
\subsubsection[normrnd]{\setlength{\rightskip}{0pt plus 5cm}double normrnd (void)}}
\label{randgen_8cpp_a18}


Gaussion random variable with zero mean and variace 1.0. Taken from Numerical Cecipies in C.\hypertarget{randgen_8cpp_a16}{
\index{randgen.cpp@{randgen.cpp}!rseed@{rseed}}
\index{rseed@{rseed}!randgen.cpp@{randgen.cpp}}
\subsubsection[rseed]{\setlength{\rightskip}{0pt plus 5cm}void rseed (long {\em idum})}}
\label{randgen_8cpp_a16}


Set the seed of the random number generator. \hypertarget{randgen_8cpp_a15}{
\index{randgen.cpp@{randgen.cpp}!uniform_rand@{uniform\_\-rand}}
\index{uniform_rand@{uniform\_\-rand}!randgen.cpp@{randgen.cpp}}
\subsubsection[uniform\_\-rand]{\setlength{\rightskip}{0pt plus 5cm}double uniform\_\-rand (long $\ast$ {\em idum})}}
\label{randgen_8cpp_a15}


Long period ($>$ 2 10 18 ) random number generator of L'Ecuyer with Bays-Durham shuffle and added safeguards. Returns a uniform random deviate between 0.0 and 1.0 (exclusive of the endpoint values). Call with idum a negative integer to initialize; thereafter, do not alter idum between successive deviates in a sequence. RNMX should approximate the largest floating value that is less than 1. \hypertarget{randgen_8cpp_a17}{
\index{randgen.cpp@{randgen.cpp}!unirnd@{unirnd}}
\index{unirnd@{unirnd}!randgen.cpp@{randgen.cpp}}
\subsubsection[unirnd]{\setlength{\rightskip}{0pt plus 5cm}double unirnd (void)}}
\label{randgen_8cpp_a17}


Returns a random number from the interval (0,1). 